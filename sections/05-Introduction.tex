\newpage
%\section{Introduction}
\chapter{Introduction}
\label{sec:introduction}
\section{Motivation}






\newpage
\subsection{Project report overview}
The following chapters in this project report delve into the design of a miniature camera on the Disruptive Technology sensor platform. Different image sensors will be considered, with respect to design parameters such as size, battery lifetime, energy, power dissipation and area of application. Three different system alternatives with different system partitioning are analyzed and evaluated. One system is proposed for implementation.\\  

Chapter \ref{sec:theory} is the Theory section. 


Chapter \ref{sec:methodology} is the Methodology section. 


Chapter \ref{sec:results} is the Results section. Results from estimations of energy consumption, battery-capacity consumption and power dissipation are presented here based on methodology presented in chapter \ref{sec:methodology}. Also, experimental results from doing image processing on pictures taken with the NanEye2D image sensor are presented here. 
\\

Chapter \ref{sec:Discussion} is the Discussion section. The results of the energy and power estimations for the three system alternatives are discussed here. A comparison of the different alternatives is made based on metrics such as number of frames possible to capture, process and transmit, area of application and compression rate possible. 
\\

Chapter \ref{sec:conclusion} is the Conclusion section. The most important results are presented here. The proposed system of the miniature camera is presented as well. Future work that can be done is mentioned, listed in bullet-points. 
\\




