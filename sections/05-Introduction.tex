\newpage
%\section{Introduction}
\chapter{Introduction}
\label{sec:introduction}
\section{Motivation}

This master thesis is part of the NTNU SmallSat \cite{SmallSat_project_description} project. The projects mission is to use hyperspectral imaging to observe ocean color in the ocean and the coast of Norway. The payload of the satellite will be a 1/3 U push-broom type hyperspectral imager, dedicated to take images of a $30x50 km^2$. In regular Red Green Blue(RGB) imaging each of the image pixels is made up of three frequency components that represent the intensities in red, green and blue frequencies respectively. Such a component is referred to as a band. In hyperspectral imaging, a pixel will typically consists of hundreds to thousands of bands, providing more information than regular images. This information can be used for a lot of different purposes. It can for example be used to detect different materials in an area, by using spectral signatures of materials as identifiers. This is shown in Figure \ref{fig:HSI_concept}.\\

\begin{figure}[H]
\centering
   \includegraphics[scale=0.3]{images/Imaging-Spectroscopy-Concept.png}
  \caption{ Functional concept of HSI.\cite{HSI_concept} } 
  \label{fig:HSI_concept}
\end{figure}

The hyperspectral imager used in NTNU SmallSat project has 100 usable bands, with a sensor resolution of 2048 x 1088 pixels. The number of effective pixels, $N_{pixels}$, is 578. This is the number of effective pixels per line/row in the image. The hyperspectral imagers operation can be seen in Figure \ref{fig:HSI_functionality}. 
\begin{figure}[H]
\centering
   \includegraphics[scale=0.3]{images/hyperspectral_imager.PNG}
  \caption{ Push-broom hyperspectral imager mode of operation. The imager captures data one pixel at the time, in a row-wise fashion \cite{SmallSat_project_description}. } 
  \label{fig:HSI_functionality}
\end{figure}
\\

NTNU SmallSat aims to use this information to detect algae blooms, phytoplankton, oil spills, microplastic and possibly other irregularities or \textit{anomalies} in the ocean. Detection of harmful algae blooms (HABs) is particularly interesting for the salmon farms located along the coast of Norway, as such blooms can be toxic, even deadly, for the salmon. Algaes were most likely the cause of death for 38 000 salmons in southern Troms in September of 2017 \cite{laksedeath}. An image of such a bloom can be seen in Figure \ref{fig:algae_bloom_troms}. Increasing ocean temperatures as a consequence of global warming may lead to more frequent and intense HABs \cite{climate_change_algae_blooms}. 
%With the increasing rise in sea temperatures and the issues faced with global warming this is an even 
% HSI  = hyperspectral imager or imaging?
\\

\begin{figure}[H]
\centering
   \includegraphics[scale=0.3]{images/algaes/algaes_northern_troms.jpg}
  \caption{ Image of an algae bloom along the coast of Troms in Norway \cite{laksedeath}. Foto by NASA EARTH OBSERVATORY. } 
  \label{fig:algae_bloom_troms}
\end{figure}
\\

Algaes will have spectral signature that is different to the background, which will be ocean water or land. Algaes may therefore be considered \textit{anomalies}. An anomaly in the context of HSI is a spectral pixel vector that have significant spectral differences from its surrounding background pixels \cite{yang2015dual}.  
\\

Anomaly detection may help combat and monitor the challenges faced globally as a consequence of global warming and human pollution. One of these challenges is the vast amount of micro-plastic( plastic particles smaller than $5mm$) in the worlds oceans. Anomaly detection may be used to detect spots of ocean water having higher density of micro-plastic than the surrounding ocean water, if such spots exists.       \\


According to \cite{SmallSat_project_description} the time requirements for actual detection of pushbroom scanning and HSI operations is 54 seconds. The goal of the Anomaly detector(AD) is to be able to operate in near-realtime or realtime, using at most 54 seconds. 


\\






\newpage
\subsection{Project report overview}
The following chapters in this project report delve into the implementation of an AD on a Xilinx Virtex-7 FPGA. \\  

Chapter \ref{sec:theory} is the Theory section.  Algorithms considered used for anomaly detection are described in this chapter.  \\


Chapter \ref{sec:methodology} is the Methodology section. In this section tests of the different anomaly-detectors has been done.  


Chapter \ref{sec:results} is the Results section. 
\\

Chapter \ref{sec:Discussion} is the Discussion section
\\

Chapter \ref{sec:conclusion} is the Conclusion section. The most important results are presented here. The proposed system of the miniature camera is presented as well. Future work that can be done is mentioned, listed in bullet-points. 
\\




