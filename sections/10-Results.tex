\newpage
%\section{Results}
\chapter{Results}
\label{sec:results}
\section{Synthesis}
\label{sec:synthesis_results}
All synthesis results in this chapter are synthesized for $Pixel\_Data\_Width$ of 16, unless another value is 
especially mentioned. Results presented in this chapter are gathered from synthesis utilization reports.\\

The designs were synthesized in Vivado. As Zynq-7000 - Z7030/Z7035 was not available for synthesis, the Zedboard Evaluation and Development kit was used for synthesis. This kit contains less logic than the Z7030/Z7035. The kit contains only 220 DSPs. This leads to the the \textbf{ACAD inverse} over-utilizing DSPs when running synthesis for designs with $P\_bands$ $>=$ 20. When over-utilizing DSPs, the logic gets mapped to LUTs instead, as described in \cite{cite:dsp_overutilizing}, and will produce unusable synthesis results. Therefore; the design was synthesized for xc7k160tiffv676-2L, as this device contains 600 DSPs, in addition to having a similar architecture as the Zedboard (has Slice Registers and Slice LUTs, as opposed to CLB Registers and CLB LUTs). \textbf{ACAD correlation} also over-utilizes DSPs for $P\_bands$ >=60 and $Pixel\_data\_width$ = 16. Therefore, \textbf{ACAD correlation} is synthesized for xc7k160tiffv676-2L for $P\_bands$ >= 60 and $Pixel\_data\_width$ = 16.\\

Timing results when synthesizing for xc7k160tiffv676-2L is not considered usable, as the device logic is different to the Zedboard. The Zedboard contains an Artix-7 device, which is a slower device than the Z-7030/Z-7035, which are Kintex-7 devices. Designs that meets timing demands for the Zedboard will therefore also most likely meet timing requirements for the Z-7030/Z-7035 devices. In addition to this, the initial test prototype is to be implemented on a Zedboard. As such, it is valuable to see if the design meet timing when running on Zedboard. Therefore, only timing results from synthesis on the Zedboard Evaluation and development kit is presented. 



\subsection{Shiftregister}
\textbf{Shiftregister} was synthesized for $P\_bands$= [10, 20, 30, 40, 50, 60, 70, 80, 90, 100]. Number of synthesized Slice registers and Slice LUTs are shown in Figure \ref{fig:primitives_shiftregister}. 

\begin{figure}[H]

\hbox{\hspace*{-2cm}                                                           
   \includegraphics[scale=0.3]{images/syntese_resultat/shiftregister.png}}
  \caption{\textbf{Shiftregister} synthesis results.  } 
  \label{fig:primitives_shiftregister}
\end{figure}

\subsection{ACAD correlation}

The design was synthesized for $P\_bands$= [10, 20, 30, 40, 50, 60, 70, 80, 90, 100]. Figure \ref{fig:primitves_correlation}  shows the number of synthesized BRAM36E1 and DSP48E1. Figure \ref{fig:luts_and_regs_corr} shows number of synthesized Slice Registers and Slice LUTs as a function of $P\_bands$.

\begin{figure}[H]

\hbox{\hspace*{-2cm}                                                           
   \includegraphics[scale=0.3]{images/number_of_BRAMS_and_DSP48_correlation_module.png}}
  \caption{Number of synthesized BRAM36E1 and DSP48E1 as a function of $P\_bands$ for the \textbf{ACAD correlation} block.  } 
  \label{fig:primitves_correlation}
\end{figure}


\begin{figure}[H]

\hbox{\hspace*{-2cm}                                                           
   \includegraphics[scale=0.3]{images/correlation_luts_and_registers.png}}
  \caption{Number of synthesized Slice Registers and Slice LUTs as a function of $P\_bands$ for the \textbf{ACAD correlation} block. } 
  \label{fig:luts_and_regs_corr}
\end{figure}

\subsubsection{$Pixel\_data\_width$ = 10}
 As \textbf{ACAD correlation} inferred a large number of DSPs, $Pixel\_data\_width$ was lowered to see if the number of DSPs inferred would be lowered. The design inferred DSPs for $Pixel\_data\_width$ >= 11, but for $Pixel\_data\_width$ = 10 the synthesis tool did not inferr any DSPs. Instead the logic was mapped to LUTs. Number of BRAMs synthesized are unchanged when varying $Pixel\_data\_width$. Number of LUTs and registers synthesized are shown in Figure  \ref{fig:correlation_luts_and_registers_10}. 
 
\begin{figure}[H]

\hbox{\hspace*{-2cm}                                                           
   \includegraphics[scale=0.3]{images/syntese_resultat/acad_correlation_using_pixel_data_with_10_luts_and_registers.png}}
  \caption{Number of synthesized Slice Registers and Slice LUTs as a function of $P\_bands$ for the \textbf{ACAD correlation} block for $Pixel\_data\_with$ =10. } 
  \label{fig:correlation_luts_and_registers_10}
\end{figure}
 
 

\subsection{ACAD inverse}

\textbf{ACAD inverse} was synthesized using the three different division-approaches.  Number of BRAMs synthesized for the three approaches are equal, as shown in Figure \ref{fig:brams_inverse}. 


\begin{figure}[H]

\hbox{\hspace*{-1cm}                                                           
   \includegraphics[scale=0.27]{images/syntese_resultat/inverse/number_of_dsps.png}}
  \caption{Number of DSP48E1 synthesized for the \textbf{Inverse} block. } 
  \label{fig:dsps_inverse}
\end{figure}


\begin{figure}[H]

\hbox{\hspace*{-1cm}                                                           
   \includegraphics[scale=0.27]{images/syntese_resultat/inverse/brams.png}}
  \caption{Number of BRAMs synthesized for the \textbf{Inverse} block. } 
  \label{fig:brams_inverse}
\end{figure}


\begin{figure}[H]

\hbox{\hspace*{-2cm}                                                           
   \includegraphics[scale=0.3]{images/syntese_resultat/inverse/number_of_luts.png}}
  \caption{Number of LUTs synthesized for the \textbf{Inverse} block. } 
  \label{fig:luts_inverse}
\end{figure}


\begin{figure}[H]

\hbox{\hspace*{-2cm}                                                           
   \includegraphics[scale=0.3]{images/syntese_resultat/inverse/number_of_registers.png}}
  \caption{Number of registers synthesized  for the \textbf{Inverse} block.. } 
  \label{fig:registers_inverse}
\end{figure}




%\subsubsection{LUTs and registers}
%\label{sec:synthesis:luts_and_registers_inverse}
%Synthesis results for the first and the second implementation approach of the Gauss-Jordan elimination is shown in Figure \ref{fig:synthesis_result_naive_inverse}.
%
%\begin{figure}[H]
%\hbox{\hspace*{-2cm}                                                           
%
%   \includegraphics[scale=0.3]{images/inverse_hw/inverse_matrix_luts_and_registers.png}}
%  \caption{Synthesis result for the first and second Gauss-Jordan implementation approach, showing number of LUTs and registers synthesized.  } 
%  \label{fig:synthesis_result_naive_inverse}
%\end{figure}

\subsection{Timing results}
To check if the design met timing, the WNS of the synthesized designs was checked. The target clock frequency was set to 100 MHz. 
\subsubsection{WNS ACAD Correlation}
\textbf{ACAD correlation} was synthesized for $P\_bands$ = [10, 20, 30, 40, 50] and $Pixel\_data\_width$ = 16 on the ZedBoard Zynq Evaluation and Development Kit. The timing results are presented in Table \ref{tab:wns_correlation}.

\begin{table}[H]
    \centering
    % \resizebox{1.\textwidth}{!}{
    \begin{tabular}{c|c}
    \textbf{$P\_bands$} &\textbf{WNS [ns]} \\
    10 & 4.235 \\
    20 & 0.721\\
    30 &1.446\\
    40 & 0.845 \\
    50 & 0.509 \\
    \end{tabular}%}
    \caption{Timing results for \textbf{ACAD correlation} $Pixel\_data\_width$ = 16.}
    \label{tab:wns_correlation}
\end{table}

\textbf{ACAD correlation} was synthesized for $P\_bands$ = [10, 20, 30, 40, 50, 60, 70, 80, 90] and $Pixel\_data\_width$ = 10 on the ZedBoard Zynq Evaluation and Development Kit. The timing results are presented in Table \ref{tab:wns_correlation_10}.
\begin{table}[H]
    \centering
    % \resizebox{1.\textwidth}{!}{
    \begin{tabular}{c|c|c|c}
    \textbf{$P\_bands$} &\textbf{WNS [ns]}& \textbf{Net delay [ns]}& \textbf{Logic delay [ns]} \\
    10 &-3.074 & 7.860 & 5.078 \\
    20 & -3.309& 7.868 & 5.305 \\
    30 & -3.109 & 6.732 & 6.241\\
    40 & -3.319 & 7.698 & 5.485 \\
    50 & -3.324 & 7.703 & 5.485 \\
    60 & -3.331 & 7.518 & 5.677 \\
    70 & -4.923 & 8.563 & 6.224 \\
    80 & -4.881 & 8.521 & 6.224 \\
    90 & -5.224 & 9.373 & 5.735 \\
    
    \end{tabular}%}
    \caption{Timing results for \textbf{ACAD correlation} $Pixel\_data\_width$ = 10.}
    \label{tab:wns_correlation_10}
\end{table}


\subsubsection{WNS division operator}
Implementing division by the use of the division operator "/" yielded the timing results presented in Table \ref{tab:division_operator_wns} when synthesizing block \textbf{Last division}, computing the product $C= B*\frac{1}{A}$. Width of $B$ is 32 bit. The design was synthesized for dividend and divisor data width of 32, 8, 6 and 5, with a target clock constraint of 100 MHz. The target device was the Zedboard Zynq Evaluation and Development kit. 
\begin{table}[H]
    \centering
     \resizebox{1.\textwidth}{!}{
    \begin{tabular}{c|c|c}
    \textbf{Data width divisor and dividend} &\textbf{WNS [ns]}&\textbf{Max frequency [MHz] } \\
         32&-80.524&11.046 \\
         8 & -11.776&45.934 \\
         6&-7.071& 58.578\\
         5 & -5.730&63.572 \\
         
    \end{tabular}}
    \caption{Synthesis results for ZedBoard Zynq Evaluation and Development Kit for \textbf{Last division} using division operator "/".}
    \label{tab:division_operator_wns}
\end{table}
\subsubsection{Worst Negative Slack Adaptive Shifting}
 The design shown in Figure \ref{fig:adaptive_shifting} was synthesized for Zedboard Zynq Evaluation and Development kit Z7020 to check timing for $P\_bands$= [10, 20, 30, 40, 50, 60]. The worst WNS was 2.034 ns, which yields $f_{max}$ of \approx 125 MHz. 
 
 %The max logic delay was $6.932$ns which gives a max operating frequency of 144.25 MHz, not accounting for net delay. 
 
 
 \subsubsection{Worst Negative Slack LUT approach}
 The \textbf{Inverse} block was synthesized for Zedboard Zynq Evaluation and Development kit 7020 with $Div\_Precision$ = 17 and $P\_bands$ =10. The synthesis results yielded a WNS of -5.972 ns. 4.847 of this is net delay. \\
 
$P\_bands$ =10 was chosen due to the LUT-approach over-utilizing DSP48E1 for higher number of $P\_bands$. 
 
 
 \section{Simulation}
% The designs have been tested on simulation runs in Vivado.  
\subsection{Shiftregister}
\textbf{Shiftregister} has been simulated and tested for constrained random inputs of $din$, and for values of $P\_bands$ dividable by 4, satisfying the condition $modulo(P\_bands,4)=0$. Figure \ref{fig:simulation_shiftregister} shows a simulation of $P\_bands$ = 12. 


\begin{figure}[H]

\hbox{\hspace*{-2.3cm}                                                           
   \includegraphics[scale=0.45]{images/simulation_results/shiftregister_p_bands_12.PNG}}
  \caption{simulation of \textbf{Shiftregister} for $P\_bands$ = 12. } 
  \label{fig:simulation_shiftregister}
\end{figure}
 
 \subsection{ACAD correlation}

 Constrained random input simulation of \textbf{ACAD correlation} has been done in Vivado, and the captured waveforms has been visually inspected. The waveforms shown in Figure \ref{fig:simulation_results_correlation_one} and \ref{fig:simulation_results_correlation_two} shows simulations of input pixel vectors of size $P\_bands$ =4, $Pixel\_data\_width$ =16.\\
 
 Figure \ref{fig:simulation_results_correlation_one} shows a simulation for a data input pixel vector of [0x00ff, 0x00f9, 0x00a5, 0x0055]. This is simulated to be the first pixel of the hyperspectral image.\\
 
 Figure \ref{fig:simulation_results_correlation_one} shows a simulation for a data input pixel vector of [0x0015, 0x00aa, 0x0029, 0x0009]. This is simulated to be the second pixel of the hyperspectral image.
 
 



\begin{figure}[H]
\refstepcounter{figure}
\begin{tabular}{c|c}

   \includegraphics[scale=0.6, angle=90, origin=c]{images/simulation_results/correlation_pixel_one.PNG}
   \rotatebox[origin=c]{90}{ Figure~\thefigure: Simulation of the \textbf{ACAD correlation} block.}
  %\caption{ \textbf{ROTATE}FSM controlling the architecture shown in Figure  } 
  \end{tabular}
  \label{fig:simulation_results_correlation_one}
\end{figure}


\begin{figure}[H]
\refstepcounter{figure}
\begin{tabular}{c|c}

   \includegraphics[scale=0.5, angle=90, origin=c]{images/simulation_results/correlation_pixel_two.PNG}
   \rotatebox[origin=c]{90}{ Figure~\thefigure: Simulation of the \textbf{ACAD correlation} block.}
  %\caption{ \textbf{ROTATE}FSM controlling the architecture shown in Figure  } 
  \end{tabular}
  \label{fig:simulation_results_correlation_two}
\end{figure}


 
 \subsection{Inverse}
 Simulation has been done with $P\_bands$=[4,6] and $Pixel\_data\_width$= 16, with constrained random input.  Figure \ref{fig:simulation_results_inverse} shows a simulation with $P\_bands$ = 4.   


\begin{figure}[H]
\refstepcounter{figure}
\begin{tabular}{c|c}

   \includegraphics[scale=0.6, angle=90, origin=c]{images/simulation_results/complete_inverse_bram_approach_4_by_4_matrix_formatted_for_latex.png}
   \rotatebox[origin=c]{90}{ Figure~\thefigure: Simulation of the \textbf{Inverse} block.}
  %\caption{ \textbf{ROTATE}FSM controlling the architecture shown in Figure  } 
  \end{tabular}
  \label{fig:simulation_results_inverse}
\end{figure}

