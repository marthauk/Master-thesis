\newpage
\chapter{Conclusion}
\label{sec:conclusion}


%%Future Work
%Future of FPGAs? Scaling, Moores law...
%MCU\\
%Battery-technology?\\
%Lumen -measurement on chip?(For image processing)\\

%Are your results satisfactory?\\
%Can they be improved?\\
%Is there a need for improvement?\\
%Are other approaches worth trying out?\\
%Will some restriction be lifted?\\
%Will you save the world with your Nifty Gadget?++
%
A proposed implementation of the ACAD algorithm has been made. This algorithm was chosen after a comprehensive review of existing AD algorithms. The ACAD algorithm has been tested on real and synthetic image data by the author and by Chang \cite{chang2006characterization}, and shows promising results. \\

The causality of the ACAD algorithm is beneficial for hardware implementation, as it enables real-time anomaly detection. ACAD builds a binary anomaly map. It is then possible to send the binary anomaly map instead of transmitting $\delta^{ACAD}(\textbf{x}_{k})$. This is advantageous with regards to data transmission, as this lowers transmission time. Transmission cost is  \\

The main bottleneck of the processing pipeline in the ACAD is the inverse computation. 

\section{Future work}

\subsection{Optimizations}


\\ 
Completing the ACAD anomaly detector. The \textbf{dACAD} should be implemented, as well as \textbf{FSM ACAD}. When the ACAD anomaly detector is completed, it should be tested on a Zedboard Zynq Evaluation and Development kit.



Set the value $\tau$ based on experimental results. The experiments should contain real hyperspectral image data from coastal areas with algae that are interesting for the SmallSat project. $\tau$ is important to be able to make a correct anomaly map. \\

Inverse computation; use Matrix Inversion Lemma as proposed by Hsueh in \cite{hsueh_master_thesis}.\\

Power optimization should be done. This is especially important as the ACAD anomaly detector is to be implemented on an energy-limited satellite. One of the most efficient and easiest power optimization is the usage of clock enable signals for sub-modules in the design; for instance \textbf{ACAD inverse}, \textbf{ACAD correlation}, \textbf{dACAD}. Their respective sub-modules could also have clock enable signals.\\

Precision LUT approach must be checked!\\

Verification of the design must be done. As of now the only form of verification done has been constrained random input simulation on the blocks \textbf{Shiftregister}, \textbf{ACAD correlation} and \textbf{ACAD inverse}. The designs should be further simulated for a wider range of inputs. An automatic test-setup should be made, with a golden reference model, possibly by using MATLAB or other high-level tools or languages. The design should also be tested on the Zedboard Evaluation and Development kit. 
