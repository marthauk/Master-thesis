\newpage

\Huge{\textbf{Project Assignment}}
\\
\\
\\
\small
\textbf{Candidate name: }  Martin Haukali
\\
\\
\\
\textbf{Assignment title:}  FPGA Implementation of Hyperspectral Anomaly Detection Algorithm
\\
\\
\textbf{ Assignment text: }
This topic is part of the large project Hyperspectral Imaging in Small Satellites. Hyperspectral imaging relies
on sophisticated acquisition and on data processing of hundreds or thousands of image bands. Most of the
algorithms for hyperspectral imaging perform intensive matrix manipulations, and
FPGAs are recommended to be used due to reconfiguration, low consumption, compact size and high
computing power.

Anomaly detection is an important task for hyperspectral data exploitation.
A standard approach for anomaly detection in the literature is the method called RX algorithm.
The computational cost is very high for RX algorithm and current advances
in high performance computing can be good solution to reduce the run- time of this algorithm.\\

Tasks:\\
- Optimization of RX algorithm for parallel processing\\
- Covariance computation\\
- Hardware implementation of inverse matrix problem\\
- FPGA implementation of RX algorithm \\


\textbf{Supervisor:  }Kjetil Svarstad 
\\
\\
\textbf{Co-supervisor: }Milica Orlandic